\section{演習問題}
\subsection{問1,2}
問題文が解説なので省略。わからなければ誰でもいいので先輩に聞こう。
\subsection{問3}
\lstinputlisting{\codepath/sakaki/answer3.c}
\subsubsection{解説}
\begin{verbatim}
非常にメンドクサイ問題である。まず、問題点を確認する。
・与えられる文字列の長さがわからない
・1文字飛ばして出力するので、改行やnull文字になるまで出力では出来ない。
よって、与えられた文字列の長さを調べる関数を用いて、その長さの中で文字を1つ飛ばしで出力すればよい。
str_len関数がそれである。この関数で、実際に入力された文字列の長さを調べている。
ここまでできればfor文で今まで通りに書けば出来る。
確かstring.hにこんな関数があったと思うけど......。
\end{verbatim}
\subsection{問4}
\lstinputlisting{\codepath/sakaki/answer4.c}
\subsubsection{解説}
問2で大雑把には説明したが、文字列を受け取り、その文字が'a'から'z'ならば-32,'A'から'Z'ならば+32して出力するプログラムを書けばよい。
for文の\verb| str[i]!='\0' |は文字列の末尾でないならばという意味である。
また、printf関数の部分で見慣れないコーディングがされていると思うが、これは三項演算子と呼ばれるもので、if文を簡潔に書けるもの程度の認識で良い。(もう少し利点はあるけど。)
if文で書くならば
\begin{verbatim}
if('a'<=str[i]&&str[i]<='z'){
     printf("%c",str[i]-32);
}
else{
     printf("%c",str[i]+32);
}
\end{verbatim}
となる。かなり簡潔になっているのがわかるだろう。ただし、読みづらいと感じる人も多いので、そこはうまく使ってほしい。