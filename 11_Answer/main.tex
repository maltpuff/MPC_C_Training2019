%------------------------------------- ページサイズなどの書式設定
%¥documentclass[a4j,twocolumn, dvipdfmx]{jsarticle} % 二段組の構成にする
%¥documentclass[a4j,notitlepage]{jsarticle} % タイトルだけのページを作らない
\documentclass[a4j,titlepage,dvipdfmx]{jsarticle}   % タイトルだけのページを作る
%-------------------------------------コマンド定義
%styファイルのパスの簡略化
\newcommand{\stypath}{./sty}
%コードファイルの簡略化(./code/04のように毎回変更する)
\newcommand{\codepath}{./code}
%記事ファイルの簡略化(codepathと同様)
\newcommand{\articlepath}{./article}
%------------------------------------- パッケージ読み込み
\usepackage[ipaex]{pxchfon}
%\usepackage{itembkbx}
\usepackage{\stypath/listings}
\usepackage{ascmac}
\usepackage{\stypath/jlisting}
\lstset{%
showstringspaces=false,%空白文字削除
language={C},% %言語選択
basicstyle={\upshape},% %標準の書体
identifierstyle={\small},% %キーワードでない文字の書体
ndkeywordstyle={\small},% %キーワードその2の書体
stringstyle={\small\ttfamily},% %””で囲まれた文字などの書体
frame={tb},% %枠、デザインなど
breaklines=true,% %行が長くなった時の自動改行
columns=[l]{fullflexible},% %書体による列幅の違いを調整するか
numbers=left,% %行番号を表示するか
xrightmargin=0zw,% %余白の調整?
xleftmargin=0zw,% %余白の調整
numberstyle={\scriptsize},%行番号の書体
stepnumber=1,% %行番号をいくつ飛ばしで表示するか
numbersep=1zw,% %行番号と本文の間隔
morecomment=[l]{//}%
}

\title{C言語講座第11回解答}%何回か書き直す
\author{MPC部員}
\date{2019年X月X日}%日付も書き直す
\begin{document}
\maketitle
\section{演習問題(さんたろー作)}
\subsection{問1}
\subsubsection{解答例}
\lstinputlisting{\codepath/papyrus/exercise1.c}
\subsubsection{解説}
\begin{verbatim}

https://papyrustaro.hatenablog.jp/entry/2018/12/30/145415
\end{verbatim}

\subsection{問2}
\subsubsection{解答例}
\lstinputlisting{\codepath/papyrus/exercise2.c}
\subsubsection{解説}
\begin{verbatim}

#defineはプログラムの記述自体を書き換えている
意味ではなく、文字列として
MULTの部分を置き換えてから、計算してみよう。
\end{verbatim}

\subsection{問3}
\subsubsection{解答例}
\lstinputlisting{\codepath/papyrus/exercise3.c}
\subsubsection{解説}
\begin{verbatim}
記事の通り。
2年後期の講義でやった内容です。
\end{verbatim}

\section{演習問題}
\subsection{問1,2}
問題文が解説なので省略。わからなければ誰でもいいので先輩に聞こう。
\subsection{問3}
\lstinputlisting{\codepath/sakaki/answer3.c}
\subsubsection{解説}
\begin{verbatim}
非常にメンドクサイ問題である。まず、問題点を確認する。
・与えられる文字列の長さがわからない
・1文字飛ばして出力するので、改行やnull文字になるまで出力では出来ない。
よって、与えられた文字列の長さを調べる関数を用いて、その長さの中で文字を1つ飛ばしで出力すればよい。
str_len関数がそれである。この関数で、実際に入力された文字列の長さを調べている。
ここまでできればfor文で今まで通りに書けば出来る。
確かstring.hにこんな関数があったと思うけど......。
\end{verbatim}
\subsection{問4}
\lstinputlisting{\codepath/sakaki/answer4.c}
\subsubsection{解説}
問2で大雑把には説明したが、文字列を受け取り、その文字が'a'から'z'ならば-32,'A'から'Z'ならば+32して出力するプログラムを書けばよい。
for文の\verb| str[i]!='\0' |は文字列の末尾でないならばという意味である。
また、printf関数の部分で見慣れないコーディングがされていると思うが、これは三項演算子と呼ばれるもので、if文を簡潔に書けるもの程度の認識で良い。(もう少し利点はあるけど。)
if文で書くならば
\begin{verbatim}
if('a'<=str[i]&&str[i]<='z'){
     printf("%c",str[i]-32);
}
else{
     printf("%c",str[i]+32);
}
\end{verbatim}
となる。かなり簡潔になっているのがわかるだろう。ただし、読みづらいと感じる人も多いので、そこはうまく使ってほしい。
\section{演習ではない問題(細川作)}
	\subsection{はじめに}
		問題文の英語はノリで書いています。
		
		動的確保の問題を作ろうとしたんですが、どうも上手くいきません。
		本当はC言語でファイル操作やら付け焼刃のテクニックで作り上げたジャッジシステムをつかったものをやろうとしましたが、
		断念しました。今回の演習は私の趣味もとい最近覚えた事です。
		
	\subsection{C - system}
		前回の問題(11回)で、パイプライン処理ということを行いました。
		端末上でコマンドを使う事で、ファイルやディレクトリの操作が可能となります。
		端末で操作する方は、ls,cat,echo,rm等は、よく使うのではないでしょうか。
		端末上で動くようなスクリプトを作っている人もいるかもしれません。
		(windows powershell内でchrome -tと打つとgoogle chromeでtwitterができるスクリプトを書いている哀れな人もいます。)
		
		このような端末上で動かすコマンドは、C言語内でもある程度同じように使うことが出来ます。
		これを使うことが出来る関数をsystem関数と言い、<stdlib.h>内で定義されています。
		返値はint型で、値の大きさはOSよって違い、
		windowsで返ってくる値とUbuntuで返ってくる値が異なる事はしばしばあります。
		以下のコードを実行して、挙動を確認してみてください。
		
		\lstinputlisting{\codepath/takerou/systemMkdir.c}
		
		どうでしょうか。実行したらhogeという空のディレクトリが作られたと思います。
		もちろん他のコマンドも行うことが出来ます。例えばUbuntuではディレクトリの名前の前に、
		「.」を付けることで隠しファイルとして機能します。これを確認する場合、ls -aと入力する必要があります。
		他にも多くのコマンドがありますので、試してみましょう。grep辺りはしんどいですが、役に立ちます。
		
		余談ですが、Cだけではなく他の言語にもsystem関数は存在します。(pythonにもあったはず)
		自分の使える言語を使って、いろんなものを触ってみてください。
		
	\subsection{CTF : Fetch unzip key from this zip file!}
		CTF(Capture The Flag)という、プログラミングとは少々毛色の違う事をやってみましょう。

		渡されたzipファイルからどこかにある解除キーを手に入れ、解凍して中のファイルを確認してください。
		端末上で行う場合、zipコマンドをネットで調べると、すぐにわかると思います。
		もしコンピューターセキュリティに興味があれば、以下のサイトからCTFをやってみましょう。
		クリックで飛ぶことが出来ますが、最初はcpawCTFが良いかと思われます。
		
		\noindent
		[1]\href{https://ctf.cpaw.site/}{cpawctf}\newline
		[2]\href{http://ksnctf.sweetduet.info/}{ksnctf}\newline
		[3]\href{https://ctf.katsudon.org/}{akictf}\newline


\end{document}
