\section{ファイル書き込み}
ファイルへの書き込み(ファイル出力)についてはいくつかの関数があります。
今回は2つ紹介します。

\subsection{fputc}
半角英数字1文字を画面に表示する関数としてputchar関数というものがありますが、それのファイル用みたいなものです。つまり、ファイルに1文字出力します。

\lstinputlisting{\codepath/11-2.c}

このプログラムを実行しても、画面上では何も起こりません。ですが、text.txtを確認するとcという文字が1文字表示されているはずです。

\subsection{fprintf}
printfのファイル出力版です。こちらはfputcとは違い1文字以上出力できます

\lstinputlisting{\codepath/11-3.c}

このプログラムも起動しても画面上では特に何も起こりません.\\
ですが、text.txt上には表示されているはずです。文字を変えたり色々試してみてください。
