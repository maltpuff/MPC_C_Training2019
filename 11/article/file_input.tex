\section{ファイル読み込み}
次は今までとは逆にファイルから文字を読み込むということをします。読み込みなのでfopenのモードはrを使います。

\subsection{fgetc}
ファイルから1文字のみ読み込む関数です。

\lstinputlisting{\codepath/11-4.c}

\subsection{fscanf}
ファイルから1行読み込む関数です。scanfのファイル版です。
注意ですがscanfと同様に、空白文字があると読み込みをやめてしまいます。

\lstinputlisting{\codepath/11-5.c}

\subsection{fgets}
文字列の時にも登場しましたがファイルでも活躍する関数です。
機能はfscanfと同様にファイル方1行読み込むという機能を持ちます
ですがfscanfとは違い空白文字も読み込むので空白で区切られた文字列などを読み込みたいときはこっちの方が使いやすいです。

\lstinputlisting{\codepath/11-6.c}

これをfor文で回すとそれぞれの行を読みとることができます。
ちなみにprintfで出力するとき改行していない理由としてはfgetsの読み取り時にテキストファイルの改行文字まで読み取っているからです。

\lstinputlisting{\codepath/11-7.c}
このときtest.txtに好きな文字を入力しておいてください。それが端末に出力されたら成功です。


