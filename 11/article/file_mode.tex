\section{ファイルの扱い}

これまでの入出力はすべて端末内で手入力していましたが、今回はファイルでの入出力を行います。
(本日の内容はwandboxやideoneなどオンライン上のコンパイラではできません。なのでgccかbcpad,visual studio等を使ってください。よくわからない場合は部員に聞いてください)\\
ファイル操作を覚えるとできる事の幅が大きくなると思います。講座中にソースをいじったりして色々試してみてください。

\subsection{ファイルの開閉}
プログラムでファイルを取り扱う場合、ファイルを開いてから読み書きを行い、ファイルを閉じるという一連の流れがあります。
ファイルを開く際にはfopen関数、閉じる際はfcloseを使います。
また、fopen関数は返り値にファイルの情報を持ったファイル構造体の変数のアドレスを返します。そのアドレスをFILE構造体のポインタに格納することでファイルの情報を管理します。

subsection{モード}
fopen関数を利用する際に第2引数にファイルを開く目的としての文字列を与えます。またその時の文字列をモードといいます。モードは6種類あり、目的に応じてその中の1つを指定します。

\begin{table}[htb]
\begin{center}
\begin{tabular}{|c|c|}

\hline
r&読み込み専用\\ \hline
w&書き込み専用\\ \hline
a&追加書き込み専用\\ \hline
r+&読み込みと書き込み\\ \hline
w+&書き込みと読み込み\\ \hline
a+&追加書き込みと読み込み\\
\hline
\end{tabular}
\end{center}
\end{table}
ここで注意なのですがr+とw+は一見同じに見えますが先ほど書いた順に処理が行われます。そのため読み込みが先に来るr+ではファイルがない場合エラーとなってしまいますが、w+は先にファイルの新規作成が行われるのでエラーとならないといった違いがあります。

\subsection{ここまでのこと}
見ただけではわからないと思いますので1例を出します。

\lstinputlisting{\codepath/11-1.c}

ちなみになにも起こりません。お使いのパソコンは正常です。

\begin{itembox}{使い方}
それぞれの関数の引数FILE型のポインタ変数=fopen(ファイル名,モード)\\
fclose(FILE型のポインタ変数)
\end{itembox}

