\section{キャスト}
変数の前に(型名)と記述すると、データの型を強引に変更することが出来ます。これをキャストといいます。


\lstinputlisting{\codepath/12-1.c}
もちろん文字列を数値に変換することもできます。
ファイル操作と組み合わせて使えるようになると様々なデータも扱えるようになると思います。
また、変換先で値が収まる場合は不変ですが、収まらない場合は変換先の型の最大値+1で割った値が出るはずです。時間があれば試してみてください。

\section{sizeof}
sizeof演算子は渡された方や変数のメモリサイズを調べるといった機能があります。
\lstinputlisting{\codepath/12-2.c}
sizeofがうまく使えればsizeof array/sizeof arry[0]とすると要素数が求められたりするので使い方次第でとても便利です。