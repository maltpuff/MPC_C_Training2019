\section{演習ではない問題(細川作)}
	\subsection{はじめに}
		問題文の英語はノリで書いています。
		
		動的確保の問題を作ろうとしたんですが、どうも上手くいきません。
		本当はC言語でファイル操作やら付け焼刃のテクニックで作り上げたジャッジシステムをつかったものをやろうとしましたが、
		断念しました。今回の演習は私の趣味もとい最近覚えた事です。
		
	\subsection{C - system}
		前回の問題(11回)で、パイプライン処理ということを行いました。
		端末上でコマンドを使う事で、ファイルやディレクトリの操作が可能となります。
		端末で操作する方は、ls,cat,echo,rm等は、よく使うのではないでしょうか。
		端末上で動くようなスクリプトを作っている人もいるかもしれません。
		(windows powershell内でchrome -tと打つとgoogle chromeでtwitterができるスクリプトを書いている哀れな人もいます。)
		
		このような端末上で動かすコマンドは、C言語内でもある程度同じように使うことが出来ます。
		これを使うことが出来る関数をsystem関数と言い、<stdlib.h>内で定義されています。
		返値はint型で、値の大きさはOSよって違い、
		windowsで返ってくる値とUbuntuで返ってくる値が異なる事はしばしばあります。
		以下のコードを実行して、挙動を確認してみてください。
		
		\lstinputlisting{\codepath/takerou/systemMkdir.c}
		
		どうでしょうか。実行したらhogeという空のディレクトリが作られたと思います。
		もちろん他のコマンドも行うことが出来ます。例えばUbuntuではディレクトリの名前の前に、
		「.」を付けることで隠しファイルとして機能します。これを確認する場合、ls -aと入力する必要があります。
		他にも多くのコマンドがありますので、試してみましょう。grep辺りはしんどいですが、役に立ちます。
		
		余談ですが、Cだけではなく他の言語にもsystem関数は存在します。(pythonにもあったはず)
		自分の使える言語を使って、いろんなものを触ってみてください。
		
	\subsection{CTF : Fetch unzip key from this zip file!}
		CTF(Capture The Flag)という、プログラミングとは少々毛色の違う事をやってみましょう。

		渡されたzipファイルからどこかにある解除キーを手に入れ、解凍して中のファイルを確認してください。
		端末上で行う場合、zipコマンドをネットで調べると、すぐにわかると思います。
		もしコンピューターセキュリティに興味があれば、以下のサイトからCTFをやってみましょう。
		クリックで飛ぶことが出来ますが、最初はcpawCTFが良いかと思われます。
		
		\noindent
		[1]\href{https://ctf.cpaw.site/}{cpawctf}\newline
		[2]\href{http://ksnctf.sweetduet.info/}{ksnctf}\newline
		[3]\href{https://ctf.katsudon.org/}{akictf}\newline

