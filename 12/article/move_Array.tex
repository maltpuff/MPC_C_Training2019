\section{動的配列・動的確保}
今回は動的配列を扱います。
動的配列は普段使っている配列(静的配列)とは違いプログラム内で配列を宣言する大きさを変更できるという機能を持っています。
具体的な例を挙げると、普段使っていた配列では、要素数Nを入力した後にary[N]と宣言することはできませんでした。(あらかじめNの初期化をしていなければならなかった)
しかし関数を使うことでそれが可能になります。\\
\\

今年(2019)から授業ではpythonを扱っているようなので今回の項目はそこまで覚えようとしなくていいです。
ただC言語では動的確保は面倒だと覚えて、使うときに再度調べるかんじでいいです。
ただ、次回の講座で行うリスト構造はpythonでも使うと思うのでそちらの方が重要になってきます。

\subsection{malloc}
malloc関数を使うと自由に配列をつくることが出来ます。メモリを確保できなかった場合はNULLが返ってきます。確保したメモリはfree関数で解法するのを忘れないようにしましょう。(小さなプログラムでは大丈夫ですが、規模が大きくなった時に解放していないとメモリを無駄に消費してしまいます。)

\lstinputlisting{\codepath/12-3.c}

malloc関数は引数の分だけメモリを確保するということです。
mallocの返り値は確保したメモリのアドレスですが、それの型は(coid*)という特殊な型となっています。
この型はどのポインタ変数にも代入できるのですが、C++コンパイラではキャストしないとエラーが出ます。

\subsection{calloc}
これも動的確保をする関数です。確保したメモリ領域を0で初期化するという機能を持っています。

\lstinputlisting{\codepath/12-4.c}

\subsection{realloc}
配列の大きさを後から拡張させる機能を持っています。
追加で確保、ということではなく上書きとなりますので注意してください。

\lstinputlisting{\codepath/12-5.c}