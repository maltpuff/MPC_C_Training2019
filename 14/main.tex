%------------------------------------- ページサイズなどの書式設定
%¥documentclass[a4j,twocolumn, dvipdfmx]{jsarticle} % 二段組の構成にする
%¥documentclass[a4j,notitlepage]{jsarticle} % タイトルだけのページを作らない
\documentclass[a4j,titlepage,dvipdfmx]{jsarticle}   % タイトルだけのページを作る
%-------------------------------------コマンド定義
%styファイルのパスの簡略化
\newcommand{\stypath}{./sty}
%コードファイルの簡略化(./code/04のように毎回変更する)
\newcommand{\codepath}{./code/}
%記事ファイルの簡略化(codepathと同様)
\newcommand{\articlepath}{./article}
%------------------------------------- パッケージ読み込み
\usepackage[ipaex]{pxchfon}
%\usepackage{itembkbx}
\usepackage{\stypath/listings}
\usepackage{ascmac}
\usepackage{\stypath/jlisting}
\usepackage[dvipdfmx]{hyperref}%ハイパリンク
\lstset{% 
showstringspaces=false,%空白文字削除
language={C},% %言語選択
basicstyle={\upshape},% %標準の書体
identifierstyle={\small},% %キーワードでない文字の書体
ndkeywordstyle={\small},% %キーワードその2の書体
stringstyle={\small\ttfamily},% %””で囲まれた文字などの書体
frame={tb},% %枠、デザインなど
breaklines=true,% %行が長くなった時の自動改行
columns=[l]{fullflexible},% %書体による列幅の違いを調整するか
numbers=left,% %行番号を表示するか
xrightmargin=0zw,% %余白の調整?
xleftmargin=0zw,% %余白の調整
numberstyle={\scriptsize},%行番号の書体
stepnumber=1,% %行番号をいくつ飛ばしで表示するか
numbersep=1zw,% %行番号と本文の間隔
morecomment=[l]{//}% 
} 
\title{C言語講座第14回}%何回か書き直す
\author{MPC部員}
\date{2019年11月28日}%日付も書き直す
\begin{document}
\maketitle
\section{演習問題easy}
\begin{enumerate}
\item 二つの数字が入力して四則演算の結果を表示してください.
\begin{itembox}{入力例}
\begin{verbatim}
8 4
\end{verbatim}
\end{itembox}

\begin{itembox}{実行例}
8 + 4 = 12\\
8 - 4 = 4\\
8 * 4 = 32\\
8 / 4 = 2
\end{itembox}
\\
\item 1問目のプログラムに変更を加え、足し算引き算掛け算割り算から選択して行えるようにしてください
\begin{itembox}{入力例}
\begin{verbatim}
+
8 4
\end{verbatim}
\end{itembox}

\begin{itembox}{出力結果}
\begin{verbatim}
四則演算からえらんで下さい(+ - * /)

8 + 4 = 12
\end{verbatim}
\end{itembox}

\end{enumerate}%記事を追加する
\section{演習問題(なゆた作成)}
\begin{enumerate}
\item 二つの数字が入力されます.swap関数を作成して実行した結果を表示しなさい.
\begin{itembox}{入力例}
\begin{verbatim}
2,4
\end{verbatim}
\end{itembox}

\begin{itembox}{実行例}
4,2
\end{itembox}
\item あなたはRPGのゲームを作ります.キャラクターのステータスは別のファイルとして用意します.構造体を使って適当なファイルから読み込んだ人間のステータスをまとめ,表示してください.また表示形式は実行例に従ってください.
\begin{itembox}{sample.txt}
\begin{verbatim}
H:20
A:50
D:20
\end{verbatim}
\end{itembox}

\begin{itembox}{出力結果}
\begin{verbatim}
HP:20
ATK:50
DEF:20
\end{verbatim}
\end{itembox}

\item \textbf{この問題は最後に解いてください}次のプログラムをしらべながら理解し,実行せよ.
\lstinputlisting{\codepath/cURL.c}
\begin{itembox}{実行結果}
\begin{verbatim}
{"message": "Internal server error"}
\end{verbatim}
\end{itembox}
\end{enumerate}
\section*{参考文献}
\noindent
[1]昨年までの講座資料\newline
[2]\href{http://9cguide.appspot.com}{苦しんで覚えるC言語}\newline
[3]\href{https://yukicoder.me/}{yukicoder}\newline
[4]\href{http://www.c-tipsref.com/reference/string.html}{C言語関数辞典}
\end{document}
