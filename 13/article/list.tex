\section{list}
ついにデータ構造という範囲に入っていきます。\\
データ構造について\\
データ構造は簡単に言えばデータの管理方法(格納の仕方)です。\\
ソフトウェアの開発においてのデータ構造はプログラムのアルゴリズムの効率に大きく影響があります。\\
そのため、この範囲の説明はじっくりやっていきますので、わからないところはどんどん聞いてみてください。\\
講座ではリスト、キュー、スタックの3つを扱います。この3つは授業で扱うと思いますが結構複雑なので予習があるとかなり楽ができます。\\
今回はリスト構造についてのみやります。何故リストかというとpythonでの配列ではリストを使っているからです。\\
\subsection{リストとは}
データが順序付けられて並んだデータ構造のこと\\
自己参照構造体(ノード)とリスト型構造体(リスト)という2つの構造体により実現される\\
自己参照型ポインタ:本来のデータのほかに、自分自身と同じ構造体を指すポインタを持つ構造体のこと\\
言葉だけではよくわからないのでここで一度ソースを貼ります。\\
書き写すかぱっと見てから下を読み進めてください。
\lstinputlisting{\codepath/list19.c}
読んでもよくわかりませんが図でみると少しわかりやすいです。
\begin{table}[htb]
\begin{center}
\begin{tabular}{|c|c|c|c|c|c|c|c|c|c|}
\cline{1-1}\cline{2-2}\cline{4-4}\cline{6-6}\cline{8-8}\cline{10-10}
アドレス & 0 && 1 && 2 && 3 && 4 \\ \cline{1-1}\cline{2-2}\cline{4-4}\cline{6-6}\cline{8-8}\cline{10-10}
要素 & 要素1 && 要素2 && 要素3 && 要素4 && 要素5\\ \cline{1-1}\cline{2-2}\cline{4-4}\cline{6-6}\cline{8-8}\cline{10-10}
次のアドレス & *1 && *2 && *3 && *4 && null   \\ 
\cline{1-1}\cline{2-2}\cline{4-4}\cline{6-6}\cline{8-8}\cline{10-10}
\end{tabular}
\end{center}
\end{table}
このように配列に似た機能を持っています。ではなぜリストがあるのかです。\\
配列とリストの違い\\
配列は今まで通り、添え字を持った箱の連続のようなものです。\\
対してリストは要素をっ持った箱を任意でつなげたようなものです\\。
なので挿入の関数・削除の関数をつくることで箱と箱の間に要素の入った箱を追加したり、削除したりすることが簡単にできます。\\

ここで挿入と削除について書いた2例を出します
\lstinputlisting{\codepath/list19_insert.c}

\lstinputlisting{\codepath/list19_delete.c}
例えばリストの3番目を削除するとアドレスは0-1-2-4となり、その後2-4の間に挿入するとします。\\
そうすると以下のようになります。
\begin{table}[htb]
\begin{center}
\begin{tabular}{|c|c|c|c|c|c|c|c|c|c|}
\cline{1-1}\cline{2-2}\cline{4-4}\cline{6-6}\cline{8-8}\cline{10-10}
アドレス & 0 && 1 && 2 && 6 && 4 \\ \cline{1-1}\cline{2-2}\cline{4-4}\cline{6-6}\cline{8-8}\cline{10-10}
要素 & 要素1 && 要素2 && 要素3 && 要素6 && 要素5\\ \cline{1-1}\cline{2-2}\cline{4-4}\cline{6-6}\cline{8-8}\cline{10-10}
次のアドレス & *1 && *2 && *6 && *4 && null   \\ 
\cline{1-1}\cline{2-2}\cline{4-4}\cline{6-6}\cline{8-8}\cline{10-10}
\end{tabular}
\end{center}
\end{table}
ここで簡単な問題です。\\
配列とリストの参照速度について\\
・n番目の参照の速さ\\
・データの挿入、削除の速さ\\
それぞれで配列とリストどちらが早いと思いますか?


